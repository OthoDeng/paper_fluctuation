\documentclass{article}


\usepackage{arxiv}

\usepackage[utf8]{inputenc} % allow utf-8 input
\usepackage[T1]{fontenc}    % use 8-bit T1 fonts
\usepackage{hyperref}       % hyperlinks
\usepackage{url}            % simple URL typesetting
\usepackage{booktabs}       % professional-quality tables
\usepackage{amsfonts}       % blackboard math symbols
\usepackage{nicefrac}       % compact symbols for 1/2, etc.
\usepackage{microtype}      % microtypography
\usepackage{lipsum}
\usepackage{graphicx}
\graphicspath{ {./images/} }
\usepackage{natbib}
\usepackage{amsmath} 
\usepackage{float}



\title{I had no idea on title yet}


\author{
 Kaihuai Deng \\
  School of Hydrology and Water Resources\\
  Nanjing University of Information Science \& Technology\\
  Nanjing, China, 210044 \\
  \texttt{202383300365@nuist.edu.cn} \\
  %% examples of more authors
   \And
 Jun Yin \\
  School of Hydrology and Water Resources\\
  Nanjing University of Information Science \& Technology\\
  Nanjing, China, 210044 \\
  \texttt{jun.yin@nuist.edu.cn} \\
  %% \AND
  %% Coauthor \\
  %% Affiliation \\
  %% Address \\
  %% \texttt{email} \\
  %% \And
  %% Coauthor \\
  %% Affiliation \\
  %% Address \\
  %% \texttt{email} \\
  %% \And
  %% Coauthor \\
  %% Affiliation \\
  %% Address \\
  %% \texttt{email} \\
}

\begin{document}
\maketitle
\begin{abstract}


\end{abstract}


% keywords can be removed
%\keywords{First keyword \and Second keyword \and More}


\section{Introduction}\label{introduction}

% 背景 → 非对称性引入 → 理论演进 → 关键变量 → 数据方法 → 研究问题 → 意义

% 研究背景与科学问题引出
Earth's climate system operates far from thermodynamic equilibrium,
driven by differential solar heating and complex energy transfer
processes \cite{kleidon2016thermodynamic}. Within this nonequilibrium
framework, atmospheric water plays a crucial role, transitioning between
phases and redistributing energy through latent heat exchanges
\cite{marconi_fluctuationdissipation_2008}. 
% 温度-降水关系理论演化(热力学→动力学→非平衡统计物理)
Recent theoretical advances
in nonequilibrium statistical physics have provided new tools to analyze
climate variables, revealing signatures of underlying physical
principles that govern climate fluctuations beyond simple mean values
and linear correlations \cite{yin_nonequilibrium_2024}.

% Evolution of Temperature-Precipitation Understanding: From Thermodynamics to Dynamics 
The conceptual understanding of temperature-precipitation relationships
has evolved considerably over recent decades. \citeauthor{allen2002constraints}
first systematically established that global precipitation changes are
primarily constrained by atmospheric energetics rather than simple
moisture availability. Building on this foundation, \citeauthor{held2006robust}
introduced the influential ``wet-gets-wetter'' mechanism, demonstrating
how increased water vapor content under warming conditions intensifies
existing precipitation patterns. The initial thermodynamic
perspective---based primarily on Clausius-Clapeyron scaling---gradually
expanded as \citeauthor{o2009physical} demonstrated that precipitation extremes
depend on complex interactions between temperature lapse rates, upward
velocities, and temperature distributions during extreme events.

% Asymmetric Climate Fluctuations and Their Significance 
Despite these advances, a growing body of evidence demonstrates that
climate fluctuations exhibit pronounced asymmetry that conventional
equilibrium-based frameworks fail to capture
\cite{yin_nonequilibrium_2024}. Temperature and moisture distributions
show marked skewness in their probability density functions (PDFs), with
different characteristics for warming versus cooling episodes
\cite{ruff2012long}. These asymmetries reflect fundamental properties of
the climate system's response to perturbations and are particularly
evident during extreme events and rapid transitions
\cite{lucarini2012universal}.

% Total Column Water Vapor: A Key Integrative Variable
Total column water vapor (TCWV) represents an especially valuable
variable for nonequilibrium analysis because it integrates both
thermodynamic state (through Clausius-Clapeyron constraints) and dynamic
processes (through atmospheric circulation patterns)
\cite{held2006robust},\cite{o2010closely}. By analyzing TCWV fluctuations,
we can gain insights into how thermodynamic and dynamic processes
jointly determine precipitation patterns across different climate
regimes. The ERA5 reanalysis dataset, with its unprecedented temporal
resolution and extended coverage (1940-2024), provides a unique
opportunity to characterize the full probability distribution of water
vapor fluctuations rather than just their mean behavior
\cite{era5_monthly_single_levels}.

% Research Questions and Approach
This study addresses three central questions: (1) How do probability
density functions of TCWV fluctuations differ from those expected under
equilibrium conditions? (2) What is the relationship between asymmetric
TCWV fluctuations and vertical velocity patterns across different
precipitation regimes? (3) How do these statistical properties vary
across different climate regions and temporal scales? We hypothesize
that TCWV fluctuations will exhibit pronounced asymmetries consistent
with nonequilibrium behavior, particularly during extreme precipitation
events.

% Implications for Climate Understanding and Modeling
To test these hypotheses, we develop a PDF-based methodology focusing on
delta anomalies to characterize the temporal evolution of climate
statistics. This approach allows us to identify universal scaling
patterns while explicitly quantifying the relative contributions of
thermodynamic and dynamic processes. By characterizing water vapor
fluctuations from a nonequilibrium perspective, we provide a new
framework that may help resolve persistent biases in precipitation
forecasts, particularly for extreme events where equilibrium assumptions
break down. Furthermore, by explicitly separating thermodynamic and
dynamic contributions, our approach may provide new insights into how
climate change will affect precipitation patterns through shifts in both
moisture availability and atmospheric circulation, potentially enhancing
our ability to represent the full spectrum of hydrological variability
in a changing climate.


\section{Data}\label{data}

We utilize ERA5 reanalysis data obtained from the Copernicus Climate
Change Service. The dataset covers the period 1940-2024 with monthly
temporal resolution and spatial coverage of the entire globe. Key
variables analyzed include:

\begin{itemize}
\item
  Total column water vapor (TCWV)
\item
  Total precipitation (tp)
\item
  Vertical velocity (w) at multiple pressure levels (850, 500, and 250
  hPa)
\item
  Surface temperature (t2m)
\end{itemize}

For precipitation-related variables, we convert units from meters to
millimeters per month by multiplying by 1000 to facilitate
interpretation. Vertical velocity is analyzed in Pa/s, with negative
values indicating upward motion.

\subsection{Calculation}\label{anomaly-calculation}

Climate anomalies are calculated by removing the long-term
climatological mean: \[A(t,x,y) = V(t,x,y) - \langle V(x,y)\rangle_{t}\]
where \(V(t,x,y)\) represents the original variable, \(A(t,x,y)\) is the
anomaly, and \(\langle \cdot \rangle_{t}\) denotes the temporal average.
\protect\phantomsection\label{temporal-smoothing}{} We apply temporal
smoothing using multiple timescales (\(\tau = 5,10,15,20\) years) to
investigate multi-scale variability. The smoothing operation helps
isolate long-term trends and reduces high-frequency noise.
\protect\phantomsection\label{area-weighted-probability-density-function-analysis}{}
To account for the spherical geometry of Earth, we implement area
weighting using cosine of latitude: \(w(\phi) = \cos(\phi)\), where
\(\phi\) represents latitude. Delta anomalies are calculated as:
\(\Delta A(t,x,y) = A_{\mathrm{\text{smooth}}}(t,x,y) - A_{\mathrm{\text{smooth}}}(t_{0},x,y)\),
where \(t_{0}\) represents the initial time period. Area-weighted
histograms are computed to generate probability density functions for
each time period and smoothing scale.
\protect\phantomsection\label{asymmetry-analysis}{} To quantify
nonequilibrium signatures in TCWV fluctuations, we calculate asymmetry
metrics for the probability density functions:

\[A_{\text{asym}} = \int_{- \infty}^{\infty}P(\Delta x)\ln(\frac{P(\Delta x)}{P( - \Delta x)})d(\Delta x)\]

This metric measures the degree of time-reversal symmetry breaking, a
fundamental property of nonequilibrium systems. For equilibrium systems,
the fluctuations should be symmetric (\(A_{\text{asym}} = 0\)), while
nonequilibrium systems exhibit non-zero values. We further examine
conditional probabilities to investigate the coupling between variables:

\[P\left( \mathrm{\text{Total precipitation}}~|~w,\mathrm{\text{TCWV}} \right)\]

where \(w\) represents total precipitation. This allows us to separate
the contributions of dynamic and thermodynamic processes to TCWV
fluctuations.



\section{Results}\label{results}

Figure \ref{fig:pdf-tcwv} presents the PDF evolution for TCWV across multiple
timescales. The analysis reveals several key features:

\begin{figure}[H]
\centering
\includegraphics[width=1\linewidth,height=\textheight,keepaspectratio]{../ERA5SLP/fig2/combined_figure_tcwv.png}
\caption{(a) Global probability density functions (PDFs) of total column
water vapor anomalies (\(A\)) across multiple temporal averaging windows
(\(\tau\)), shown chronologically with color gradients from earlier
(lighter) to later (darker) periods; (b) the evolution of mean
(\(\mu\)), variance (\(\sigma^{2}\)), and asymmetry parameter
(\(\Delta\beta\)) for short (\(\tau = 1\) year, blue) and long
(\(\tau = 15\) years, red) averaging windows. Analysis is based on ERA5
global reanalysis data from 1940 to 2024.}
\label{fig:pdf-tcwv}{}
\end{figure}


\begin{figure}[H]
\centering
\includegraphics[ width=0.8\textwidth,keepaspectratio]{../ERA5SLP/fig6/combined_figure.png}

\caption{Rescaling and symmetry of total precipitation fluctuations. The
lines in (a) and (c) are rescaled from the PDFs of \(A\) and
\(\Delta A\) in Figs. Figure \ref{fig:pdf-tcwv} The tails of these PDFs are
linearly stretched to have a unit slope. The symmetry of the temperature
fluctuations is evaluated by the ratio of positive and negative
temperature fluctuations away from the mode against the differences of the
exponents of the tails for (b) \(A\) and (d) \(\Delta A\). Results are
based on ERA5.}
\label{fig:tp-distribution}
\end{figure}

Figure \ref{fig:tp-distribution} is not determined to be inserted.

% \section{Examples of citations, figures, tables, references}
% \label{sec:others}
% \lipsum[8] \cite{kour2014real,kour2014fast} and see \cite{hadash2018estimate}.

% The documentation for \verb+natbib+ may be found at
% \begin{center}
%   \url{http://mirrors.ctan.org/macros/latex/contrib/natbib/natnotes.pdf}
% \end{center}
% Of note is the command \verb+\citet+, which produces citations
% appropriate for use in inline text.  For example,
% \begin{verbatim}
%    \citet{hasselmo} investigated\dots
% \end{verbatim}
% produces
% \begin{quote}
%   Hasselmo, et al.\ (1995) investigated\dots
% \end{quote}

% \begin{center}
%   \url{https://www.ctan.org/pkg/booktabs}
% \end{center}


% \subsection{Figures}
% \lipsum[10] 
% See Figure \ref{fig:fig1}. Here is how you add footnotes. \footnote{Sample of the first footnote.}
% \lipsum[11] 

% \begin{figure}
%   \centering
%   \fbox{\rule[-.5cm]{4cm}{4cm} \rule[-.5cm]{4cm}{0cm}}
%   \caption{Sample figure caption.}
%   \label{fig:fig1}
% \end{figure}

% \begin{figure} % picture
%     \centering
%     \includegraphics{test.png}
% \end{figure}

% \subsection{Tables}
% \lipsum[12]
% See awesome Table~\ref{tab:table}.

% \begin{table}
%  \caption{Sample table title}
%   \centering
%   \begin{tabular}{lll}
%     \toprule
%     \multicolumn{2}{c}{Part}                   \\
%     \cmidrule(r){1-2}
%     Name     & Description     & Size ($\mu$m) \\
%     \midrule
%     Dendrite & Input terminal  & $\sim$100     \\
%     Axon     & Output terminal & $\sim$10      \\
%     Soma     & Cell body       & up to $10^6$  \\
%     \bottomrule
%   \end{tabular}
%   \label{tab:table}
% \end{table}

% \subsection{Lists}
% \begin{itemize}
% \item Lorem ipsum dolor sit amet
% \item consectetur adipiscing elit. 
% \item Aliquam dignissim blandit est, in dictum tortor gravida eget. In ac rutrum magna.
% \end{itemize}


\bibliographystyle{plainnat}  
\bibliography{manuscript}  

\end{document}

